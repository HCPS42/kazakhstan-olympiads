\documentclass[11pt,a4paper,twocolumn]{article}
\usepackage{amsfonts, amsmath, amsthm, amssymb}
\usepackage[landscape,left=1cm,right=1cm,top=1.5cm,bottom=1.5cm]{geometry}
\usepackage{arcs}
\usepackage{paralist}
\usepackage[russian]{babel}
\usepackage[utf8]{inputenc}
\usepackage{graphicx}

\usepackage{sectsty}
\allsectionsfont{\centering}

\usepackage{wrapfig}
\usepackage{layout}
\headsep=4mm
\footskip=4.5mm
\usepackage{floatflt}
\usepackage{caption}
\usepackage{multicol,lipsum}

\def\frac#1#2{\mathchoice{#1\over#2}{\hbox{\small$#1$}\over\mathstrut
    \hbox{\small$#2$}}{#1\over#2}{#1\over#2}}      


\newcounter{zadacha}
\newcommand{\z}{\par \smallskip \smallskip \noindent \refstepcounter{zadacha}%
\textbf{№\arabic{zadacha}.} }

\def\otv{\par \smallskip \noindent \textbf{Ответ: }}

\def\sol{\par \smallskip \noindent \textbf{Решение. }}
\def\solI{\par \smallskip \noindent \textbf{Первое решение. }}
\def\solII{\par  \smallskip \noindent \textbf{Второе решение. }}
\def\solIII{\par \noindent \textbf{Третье решение. }}
\def\lemmaI{\noindent \textbf{Лемма. }}
\def\lemma#1{\noindent \textbf{Лемма {#1}. }}
\def\proof{\par \noindent \textbf{Доказательство. }}
\def\q#1{\par \vspace{6pt plus 1pt minus 1pt} \noindent \textbf{№ #1.} }


\def\timek{\begin{flushright}\textit{Time allowed is 4 hours and 30 minutes \\
Each problem is worth 7 points}\end{flushright}}

\def\timer{\begin{flushright}\textit{Время работы: 4 часа 30 минут\\
Каждая задача оценивается в 7 баллов}\end{flushright}}

\def\tur#1{XXII Математическая Олимпиада «Шёлковый путь» \\
Март, 2023 год \\ #1}

\def\turk#1{XXII Silk Road Mathematical Competition \\
March 2023 \\ #1}


\def\rfig#1{\parpic[r]{\includegraphics{#1}}}

\makeatletter
\newdimen\normalleftskip \normalleftskip\z@
\newdimen\figleftskip \figleftskip\parindent
\makeatother


\newsavebox{\tempbox}

\def\side#1{

\savebox{\tempbox}{\includegraphics{#1}}%
\savebox{\tempbox}{\hskip\figleftskip%
   \begin{minipage}{\wd\tempbox}
      \usebox\tempbox
      \captionof{figure}{} \label{#1}
   \end{minipage}}
\begin{wrapfigure}{R}{\wd\tempbox}
\raisebox{0pt}[\dimexpr \ht\tempbox]{\usebox\tempbox}
\end{wrapfigure}

}

\usepackage{multicol}
\setlength{\columnsep}{1cm}

\begin{document}
\pagestyle{empty}


\newpage
\setcounter{zadacha}{0}
\subsubsection*{\tur{ }}
\timer

\z Внутри трапеции $ABCD$ $(AD \parallel BC)$  выбрана точка $M$, а внутри треугольника $BMC$ точка $N$ так, что $AM \parallel CN$, $BM \parallel DN$. Докажите, что у треугольников $ABN$ и $CDM$ площади равны.

\z Дано натуральное $n$. В клетчатом квадрате $2n\times 2n$  каждая клетка покрашена в какой-то из $4n^2$ цветов (при этом некоторые цвета могли не использоваться). \textit{Доминошкой} будем называть любой прямоугольник из двух клеток в нашем квадрате. Будем говорить, что доминошка \textit{разноцветная}, если клетки в ней разных цветов.

Пусть $k$ --- количество разноцветных доминошек среди всех доминошек в нашем квадрате. Пусть $\ell$ --- наибольшее целое число такое, что в любом разрезании квадрата на доминошки найдётся хотя бы $\ell$ разноцветных доминошек. Найдите наибольшее возможное значение выражения $4\ell - k$ по всем возможным раскраскам квадрата.

\z Пусть $p$ --- простое число. Построим ориентированный граф на $p$ вершинах, пронумерованных целыми числами от $0$ до $p - 1$. В графе проводится ребро из вершины $x$ в вершину $y$ тогда и только тогда, когда $y$ равно остатку от деления на $p$ числа $x^2 + 1$. Через $f(p)$ обозначим длину самого длинного ориентированного цикла в этом графе. Докажите, что $f(p)$ может принимать сколь угодно большие значения.

\z Пусть $\mathcal{M} = \mathbb{Q}[x, y, z]$ --- множество многочленов с рациональными коэффициентами от трех переменных. Докажите, что для любого ненулевого многочлена $P \in \mathcal{M}$ существуют такие ненулевые многочлены $Q, R \in \mathcal{M}$, что
$$
R(x^2 y, y^2 z, z^2 x)
=
P(x, y, z) Q(x, y, z)
.
$$

\bigskip
\bigskip
\bigskip


\textsl{\textbf{Внимание!} 
Так как XXII Математическая Олимпиада «Шёлковый путь» проводится в разных странах в разное время, мы вас убедительно просим \textbf{не разглашать} эти задачи и не обсуждать их (особенно по Интернету) до 25 мая 2023 года.
}

\newpage
\setcounter{zadacha}{0}
\subsubsection*{\turk{ }}
\timek

\z Point $M$ was chosen inside trapezoid $ABCD$ $(AD \parallel BC)$, and point $N$ was chosen inside triangle $BMC$ such that $AM \parallel CN$, $BM \parallel DN$. Prove that triangles $ABN$ and $CDM$ have equal areas.

\z Let $n$ be a positive integer. Each cell of a $2n\times 2n$ checkered square is painted in one of the $4n^2$ colors (some colors may be missing). We will call any two-cell rectangle in our square a \textit{domino}. We will also say that a domino is \textit{colorful} if its cells have different colors.

Let $k$ be the total number of colorful dominoes in our square. Let $\ell$ be the maximum integer such that every partition of the square into dominoes contains at least $\ell$ colorful dominoes. Determine the maximum possible value of $4\ell - k$ over all possible colorings of the square. 

\z Let $p$ be a prime number. Let's construct a directed graph on $p$ vertices, labeled with integers ranging from $0$ to $p - 1$. We draw an edge from vertex $x$ to vertex $y$ if and only if $y$ is equal to the remainder of division by $p$ of number $x^2 + 1$. Let $f(p)$ denote the length of the longest directed cycle in this graph. Prove that $f(p)$ can attain arbitrarily large values.

\z Let $\mathcal{M} = \mathbb{Q}[x, y, z]$ be the set of three-variable polynomials with rational coefficients. Prove that for any non-zero polynomial $P \in \mathcal{M}$ there exist non-zero polynomials $Q, R \in \mathcal{M}$ such that
$$
R(x^2 y, y^2 z, z^2 x)
=
P(x, y, z) Q(x, y, z)
.
$$

\bigskip
\bigskip
\bigskip
\bigskip
\bigskip
\bigskip

\textsl{\textbf{Attention!} 
Since the XXII Silk Road Mathematical Competition takes place in different countries on different dates, we ask you \textbf{not to disclose} these problems and not to discuss them publicly (especially through Internet) before May 25, 2023.}

\end{document}
